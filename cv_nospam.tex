\documentclass[11pt,a4paper]{moderncv}

\usepackage{verbatim}

% moderncv themes
\moderncvtheme[blue]{classic}                 % optional argument are 'blue' (default), 'orange', 'red', 'green', 'grey' and 'roman' (for roman fonts, instead of sans serif fonts)
%\moderncvtheme[grey]{classic}                % idem

% character encoding
\usepackage[utf8]{inputenc}

% adjust the page margins
\usepackage[scale=0.8]{geometry}

% before 1.0.0
\AtBeginDocument{\setlength{\makecvheadnamewidth}{9cm}}
% after 1.0.0
% \AtBeginDocument{\setlength{\makecvtitlenamewidth}{9cm}}
\AtBeginDocument{\recomputelengths}

\newcommand{\resitem}[1]{\item #1 \vspace{-1pt}}
\newcommand\espace{\vspace{10pt}}

% personal data
\firstname{Vincent}
\familyname{Cornet}
\title{Architecte Applicatif, Expert Java, DevOps}
% avoid spam : replace ONLY when sent to someone
\address{@ADDRESS1@}{@ADDRESS2@}
\mobile{@MOBILE@}
\email{@EMAIL@}

\nopagenumbers{}
%----------------------------------------------------------------------------------
%            content
%----------------------------------------------------------------------------------
\begin{document}
\maketitle


\section{Compétences et pratiques}


\cvline{Java}{Spring (Boot, Cloud, Web-MVC, Data, Security, Batch...), Hibernate (JPA), profiling (JMC, YourKit, MAT), REST, \textit{Domain Driven Design}}
\cvline{Test}{Junit, TestNG, Mockito, WireMock, Cucumber, JGiven, Ginkgo, Test Containers, Postman, SoapUI, \textit{Test/Behaviour Driven Design}}
\cvline{DevOps}{GCP, AWS, Azure, Kubernetes, Docker, Terraform, Ansible, ArgoCD, Vault, CI/CD, \textit{Infra as Code}, \textit{GitOps}}
\cvline{Data}{PostgreSQL, Oracle, HSQLDB/H2, Sybase, ElasticSearch, Kafka, MongoDB}
\cvline{Agile}{Kanban, Scrum}


\section{Expériences professionnelles}


\cventry{Depuis Mai 2022}{Sonepar}{Paris}{Senior Cloud Devops}{}{
  \begin{itemize}
    \resitem{\textbf{Métier :} Marketplace du groupe Sonepar.}
    \resitem{\textbf{Poste :} Membre d'une équipe Cloud de 6 personnes. Organisation du travail en Safe.}
    \resitem{\textbf{Sujets :} Infrastructure et outillage de la plateforme Spark. Design et provisioning \textit{as code} des différentes ressources cloud Azure/GCP et maintenance d'un parc applicatif avec ArgoCD (\textit{GitOps}) et un outillage spécifique centré autour d'opérateurs Kubernetes et des outils Azure. Support aux équipes de développement (CI/CD, Kubernetes/ArgoCD, Azure).}
    \resitem{\textbf{Technologies et outils :} Azure, GCP, Azure DevOps, Azure SDK (go/python), Terraform, ArgoCD, Argo Workflows, Kubernetes, Docker, KubeBuilder, Ginkgo, Golang, Python, Shell, Git.}
  \end{itemize}
}

%----------------------------------------------------------------------------------
\cventry{Sept. 2021 -- Mars 2022}{DPD Group}{Issy-les-Moulineaux (92)}{Développeur Java}{}{
  \begin{itemize}
    \resitem{\textbf{Métier :} Système de tracking des colis du groupe DPD.}
    \resitem{\textbf{Poste :} Membre d'une équipe de 10 personnes. Organisation du travail en Scrum.}
    \resitem{\textbf{Sujets :} Centralisation du tracking des colis pour toutes les business units du groupe, agrégeant les événements issus de ces différentes entités pour les normaliser, les stocker et les rendre exploitables. Conception et réalisation; efforts d'amélioration de la qualité générale du code (tests, pratiques).}
    \resitem{\textbf{Technologies et outils :} Spring Boot, Spring Kafka, SLF4j + Logback, Cucumber, Cassandra, Kafka, ELK, Docker, Ansible, Vault, Git, Maven, GitLabCI.}
  \end{itemize}
}

%----------------------------------------------------------------------------------
\espace

\cventry{Sept. 2020 -- Août 2021}{PMU}{Issy-les-Moulineaux (92)}{Architecte}{}{
  \begin{itemize}
    \resitem{\textbf{Métier :} Système de gestion des promotions et de la fidélité au Pari Mutuel Urbain.}
    \resitem{\textbf{Poste :} Architecte applicatif et membre d'une équipe de 5 personnes. Organisation du travail en Scrum/SAFe.}
    \resitem{\textbf{Sujets :} Refonte du back-end de gestion des promotions et de la fidélité, adaptant un produit tiers (en SaaS) au reste du SI PMU. Architecture et réalisation des différents composants applicatifs concrétisant cette interface. Ateliers d'architecture et divers travaux transverses (métrologie, logs).}
    \resitem{\textbf{Technologies et outils :} Spring Boot, Hibernate, Swagger, Test Containers, SLF4j + Logback, AWS, PostgreSQL 11, Kafka, ELK, Elastic APM, Docker, Git, Maven, GitLabCI.}
  \end{itemize}
}

%----------------------------------------------------------------------------------
\espace

\cventry{Juin 2019 -- Juillet 2020}{Renault/Renault Digital}{Plessis-Robinson/Boulogne (92)}{Référent DevOps}{}{
  \begin{itemize}
    \resitem{\textbf{Métier :} Divers projets (progiciels internes, web apps, big data...) au sein de Renault et Renault Digital.}
    \resitem{\textbf{Poste :} Référent DevOps auprès des équipes projets.}
    \resitem{\textbf{Sujets :} Accompagnement et support des équipes projets pour les aspects infrastructure (provisioning de ressources AWS/GCP, bonnes pratiques, infra as code) et livraison (conteneurisation, pipelines CI/CD, dimensionnement). Maintenance des éléments d'infrastructure, outils et middlewares mutualisés.}
    \resitem{\textbf{Technologies et outils :} AWS, GCP, Terraform 0.12, Git, GitLabCI, Docker, Kubernetes (GKE), Helm 2 et 3, Swarm, Maven, Ansible, Jenkins, Vault.}
  \end{itemize}
}

%----------------------------------------------------------------------------------
\espace

\cventry{Janv. 2015 -- Mai 2019}{Ritmx}{Paris}{Architecte Java DevOps}{}{
  \begin{itemize}
    \resitem{\textbf{Métier :} Configuration et distribution des titres de transport TER.}
    \resitem{\textbf{Poste :} Membre de différentes équipes produits et infrastructure de Ritmx (4 à 10 personnes). Méthodologies agiles (Scrum, Kanban) et DevOps. Membre du comité d'architecture qui implémente le rôle de CTO.}
    \resitem{\textbf{Sujets :} Devis et matérialisation des titres de transport (API REST). Migration sur une infra Cloud (GCP). Adoption de Kubernetes et de pratiques DevOps : \textit{Infra as Code}, CI/CD, gestion des configs et des secrets, outillage.}
    \resitem{\textbf{Technologies et outils :} Spring Boot 1 puis 2, Spring Cloud Config, Hibernate, Swagger, SLF4j + Log4j2, Fitnesse, JGiven, Tomcat 7 à 9, PostgreSQL 9.5 à 10, Git, Maven, Gradle, Jenkins, GitLabCI, YourKit, JMC, Docker, Ansible, Terraform, Packer, Vault, Kubernetes, gcloud, Prometheus, Grafana, EFK, fluent-bit.}
  \end{itemize}
}

%----------------------------------------------------------------------------------
\espace

\cventry{Sept. 2013 -- Août 2014}{Mirakl}{Paris}{Développeur Java}{}{
  \begin{itemize}
    \resitem{\textbf{Métier :} Gestion de places de marché en SaaS.}
    \resitem{\textbf{Poste :} Membre d'une équipe de 10 personnes. Organisation du travail en Scrum.}
    \resitem{\textbf{Sujets :} Restructuration de l'application (migration de Grails vers Spring \textit{vanilla}). Résolution de problèmes de performances (profiling). Soutien technique et aide à la conception auprès des autres membres de l'équipe.}
    \resitem{\textbf{Technologies et outils :} Grails 2.3, Spring 3 (AOP, ORM, Security, Batch), Hibernate 3, Hibernate Validator 5.0, Mysema QueryDSL, Jersey, Logback, JUnit puis TestNG, Mockito, Tomcat 7, PostgreSQL 9.2 puis 9.3, Git, Maven, Jenkins, MemoryAnalyzerTool, JProfiler, YourKit.}
  \end{itemize}
}

%----------------------------------------------------------------------------------
\espace

\cventry{Juin 2011 -- Août 2013}{Fia-Net}{Paris}{Architecte J2EE}{}{
  \begin{itemize}
    \resitem{\textbf{Métier :} Kwixo, solution de paiement en ligne du Crédit Agricole.}
    \resitem{\textbf{Poste :} Développement et encadrement technique dans une équipe de 12 personnes.}
    \resitem{\textbf{Sujets :} Suivi de production en collaboration étroite avec les opérationnels et DBA. Travaux de fond sur le socle technique des applications (AspectJ + LTW, API Criteria de JPA). Création et promotion de tests d'acceptance auprès du service qualité.}
    \resitem{\textbf{Technologies et outils :} Spring 3 (Web-MVC, AOP, ORM, WebFlow, Security, Batch, JMS), Hibernate 3 puis 4, Freemarker, Sitemesh, Drools, Logback, jQuery, AngularJS, JUnit, Mockito, Watir (JRuby), Concordion, Tomcat 6 puis 7, Sybase 15 puis 15.7, CouchDB 1.1, ActiveMQ 5.3 à 5.8, Git, Maven, Hudson/Jenkins, MemoryAnalyzerTool.}
  \end{itemize}
}

%----------------------------------------------------------------------------------
\espace

\cventry{Janv. 2011 -- Juin 2011}{Vidal}{Issy-les-Moulineaux (92)}{Développeur Java}{}{
  \begin{itemize}
    \resitem{\textbf{Métier :} Gestion documentaire et outillage des scientifiques et rédacteurs du Vidal.}
    \resitem{\textbf{Poste :} Membre de l'équipe back-office (4-5 personnes), dans le contexte d'une refonte globale du SI. Organisation agile du travail fondée sur Scrum.}
    \resitem{\textbf{Sujets :} Génération et transformation de documents structurés. Pose des bases d'une nouvelle GED. Effort de cartographie du SI et de documentation en général.}
    \resitem{\textbf{Technologies et outils :} Java 6, Spring 2.5, Hibernate 3, Eclipse RCP, MS SQL Server 2005/2008, Junit, Mockito, Scenari, Maven 3 (+ plugin Tycho), OSGi, SVN/git-svn, Documentum 5, Nuxeo DM, XSLT, XMLMind.}
  \end{itemize}
}

%----------------------------------------------------------------------------------
\espace

\cventry{Janv. 2010 -- Déc. 2010}{Oalia}{Suresnes (92)}{Ingénieur R\&D}{}{
  \begin{itemize}
    \resitem{\textbf{Métier :} Éditeur de logiciel d'achats.}
    \resitem{\textbf{Poste :} Ingénieur Recherche et Développement (équipe de 3 personnes).}
    \resitem{\textbf{Sujets :} Conception et développement d'éléments applicatifs divers. Refactoring continu dans un contexte technique souffrant d'obsolescence. Impulsion de divers chantiers transverses : intégration continue, tests automatisés, performances.}
    \resitem{\textbf{Technologies et outils :} Java 5 et 6, Apache Turbine, Hibernate 3, Velocity, EHCache, Quartz, Digester, EasyMock, PowerMock, Axis 1, Oracle 10g, postgreSQL, Ant, Eclipse, Jmeter, Jprofiler, SVN/git-svn.}
  \end{itemize}
}

%----------------------------------------------------------------------------------
\espace

\cventry{Nov. 2009}{Orsys}{La Défense (92)}{Formateur}{}{
  \begin{itemize}
    \resitem{\textbf{Métier :} Formation.}
    \resitem{\textbf{Poste :} Animateur d'une semaine de formation au framework Spring.}
    \resitem{\textbf{Sujets :} Présentation des concepts d'IOC, de TDD, de MVC, ainsi que des principaux modules constituant le framework Spring : IoC, AOP, ORM, Tx, MVC, Security, Remoting. Création d'exercices pour les séances de TDs, pendant lesquelles les solutions proposées par Spring sont mises en œuvre et commentées.}
    \resitem{\textbf{Technologies et outils :} Java 5, Spring Framework 3, Hibernate 3, JUnit, Maven 2, Eclipse, Open Office.}
  \end{itemize}
}

%----------------------------------------------------------------------------------
\espace

\cventry{Oct. 2009 -- Nov. 2009}{Weka}{Paris}{Développeur Java}{}{
  \begin{itemize}
    \resitem{\textbf{Métier :} Fourniture d'articles et ouvrages scientifiques.}
    \resitem{\textbf{Poste :} Intervention suite à des indisponibilités régulières en production.}
    \resitem{\textbf{Sujets :} Reproduction de l'anomalie de production dans un environnement dédié. Mesure de l'évolution des performances sous la charge, profiling. Configuration fine de différents mécanismes de cache.}
    \resitem{\textbf{Technologies et outils :} Weblogic, JMeter, ab, YourKit, Spring Framework, EHCache, Apache Jackrabbit, Eclipse, Oracle 10g.}
  \end{itemize}
}

%----------------------------------------------------------------------------------
\espace

\cventry{Mars 2008 -- Juil. 2009}{Direct-Énergie}{Issy-les-Moulineaux (92)}{Chef de projet technique}{}{
  \begin{itemize}
    \resitem{\textbf{Métier :} Distribution de gaz et d'électricité.}
    \resitem{\textbf{Poste :} Encadrement technique d'une équipe de 4 à 6 personnes.}
    \resitem{\textbf{Sujets :} Stabilisation, modularisation et évolution des applications relatives à la gestion des souscriptions. Efforts d'industrialisation pour divers projets (builds, qualimétrie, CI).}
    \resitem{\textbf{Technologies et outils :} Java 5 puis 6, Spring 2.5, Hibernate 3 + annotations, XFire/CXF, Ant puis Maven 2, JUnit/TestNG + EasyMock, log4j/slf4j, Continuum/CruiseControl puis Hudson, Archiva puis Nexus, Oracle 10g puis 11g, Jboss 4.2.2.GA puis 5.1.0.GA, activeMQ, Jira, Eclipse, SoapUI.}
  \end{itemize}
}

%----------------------------------------------------------------------------------
\espace

\cventry{Nov. 2005 -- Mars 2008}{AOL}{Neuilly-sur-Seine (92)}{Développeur Java / Lead developer}{}{
  \begin{itemize}
    \resitem{\textbf{Métier :} Portails web thématiques.}
    \resitem{\textbf{Poste :} Développement puis encadrement technique d'une équipe de 8 à 12 personnes.}
    \resitem{\textbf{Sujets :} Refonte intégrale des portails AOL France. Framework web maison fondé sur Spring, modèle de contenus enrichi (fonctionnalités type web 2.0). Ingestion de flux partenaires.}
    \resitem{\textbf{Technologies et outils :} Java 4 puis 5, Spring Framework 1 puis 2, Hibernate 3 + annotations, Acegi, JMS, Velocity puis Freemarker, Digester, Quartz, Maven 1 puis 2, jUnit, EasyMock, Log4j, Continuum, Sybase puis Oracle 10g, Tomcat 5.5, Eclipse.}
  \end{itemize}
}

%----------------------------------------------------------------------------------
\espace

\section{Diplômes et formations}

\cventry{2020}{Licence en sociologie}{}{}{}{}
\cventry{2018}{Kubernetes Application Developer (Zenika)}{}{}{}{}
\cventry{2008}{Sun Certified Programmer for the Java 2 Platform, SE 5.0}{}{}{}{}
\cventry{2005}{Ingénieur École Centrale de Lille}{}{}{}{}
\cventry{2005}{Master Recherche en informatique}{}{}{}{}
\cventry{2000}{BAC Série S}{}{}{}{}


\section{Langues}

\cvline{Français}{natif}
\cvline{Anglais}{professionnel}
\cvline{Allemand}{notions}
\cvline{Japonais}{notions}


\section{Centres d'intérêt et activités}

\cvline{Sport}{Wushu (Tai Chi)}
\cvline{Musique}{Pratique de la flûte}
\cvline{Lecture}{Philosophie, sociologie et bandes dessinées}


\end{document}

